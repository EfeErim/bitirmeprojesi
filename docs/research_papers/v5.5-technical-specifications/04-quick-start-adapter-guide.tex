

\documentclass[11pt,a4paper]{article}
\usepackage[utf8]{inputenc}
\usepackage[T1]{fontenc}
\usepackage{amsmath,amssymb,amsfonts}
\usepackage{graphicx}
\usepackage{booktabs}
\usepackage{hyperref}
\usepackage{geometry}
\usepackage{listings}
\usepackage{xcolor}
\usepackage{caption}
\usepackage{float}
\usepackage{enumitem}

\geometry{margin=2.5cm}

\definecolor{codegreen}{rgb}{0,0.6,0}
\definecolor{codegray}{rgb}{0.5,0.5,0.5}
\definecolor{codepurple}{rgb}{0.58,0,0.82}
\definecolor{backcolour}{rgb}{0.95,0.95,0.92}

\lstdefinestyle{mystyle}{
    backgroundcolor=\color{backcolour},   
    commentstyle=\color{codegreen},
    keywordstyle=\color{magenta},
    numberstyle=\tiny\color{codegray},
    stringstyle=\color{codepurple},
    basicstyle=\ttfamily\footnotesize,
    breakatwhitespace=false,         
    breaklines=true,                 
    captionpos=b,                    
    keepspaces=true,                 
    numbers=left,                    
    numbersep=5pt,                  
    showspaces=false,                
    showstringspaces=false,
    showtabs=false,                  
    tabsize=2
}

\lstset{style=mystyle}

\title{\textbf{AADS-ULoRA v5.5--Independent Multi-Crop Continual Learning Architecture}\\Quick Start Adapter Guide\\Continuous Learning for Multi-Crop Agricultural Disease Detection\\With Independent Adapters, DoRA, SD-LoRA, CONEC-LoRA, and Dynamic OOD Detection}
\author{Agricultural AI Development Team}
\date{March 2026--Version}

\begin{document}

\maketitle

\begin{abstract}
Quick start guide for AADS-ULoRA v5.5 Independent Multi-Crop Continual Learning Architecture. This guide provides a streamlined 12-week timeline for implementing practical agricultural diagnostic capabilities: Simple Crop Router $\rightarrow$ Independent Crop Adapters $\rightarrow$ Per-Crop Continual Learning with Dynamic OOD Detection. Building on v5.4's independent adapters, v5.5 introduces dynamic per-class Mahalanobis thresholds for enhanced novelty detection. Each crop maintains its own lifecycle (Base $\rightarrow$ CIL $\rightarrow$ DIL) without requiring coordination with other crops, while automatically adapting OOD detection to each disease class's variability.
\end{abstract}

\tableofcontents
\newpage

\section{When to Use Independent Multi-Crop Architecture}

\subsection{Use v5.5 If:}

\begin{itemize}[leftmargin=*]
    \item[$\checkmark$] You need multiple crops with independent update schedules
    \item[$\checkmark$] You want to add new diseases to one crop without affecting others
    \item[$\checkmark$] You cannot store all historical training data (rehearsal-free)
    \item[$\checkmark$] You want simple, practical deployment over theoretical complexity
    \item[$\checkmark$] You need proven, literature-backed continual learning methods
    \item[$\checkmark$] You want automatic OOD threshold adaptation (no manual tuning)
    \item[$\checkmark$] You need per-class OOD sensitivity (different thresholds per disease)
\end{itemize}

\subsection{Use v5.4 If:}

\begin{itemize}[leftmargin=*]
    \item[$\times$] You prefer fixed, manually-tuned OOD thresholds
    \item[$\times$] You have very limited validation data for threshold computation
\end{itemize}

\subsection{Use v5.3 If:}

\begin{itemize}[leftmargin=*]
    \item[$\times$] You need cross-crop knowledge transfer (e.g., tomato $\rightarrow$ pepper)
    \item[$\times$] You have strong ML background and want cutting-edge research
    \item[$\times$] You can invest 6+ months in complex global orchestration
\end{itemize}

\section{12-Week Implementation Timeline}

\begin{table}[H]
\centering
\caption{v5.5 Independent Multi-Crop 12-Week Timeline}
\begin{tabular}{clp{8cm}}
\toprule
\textbf{Week} & \textbf{Phase} & \textbf{Activities} \\
\midrule
1-2 & Setup & Environment, DINOv3 access, crop router training \\
3-4 & First Crop (Tomato) & Phase 1 base training with DoRA, compute prototypes and dynamic OOD thresholds, validate $\geq$95\% accuracy \\
5 & Phase 2 (CIL) & Add new disease to tomato via SD-LoRA, validate retention $\geq$90\%, update OOD thresholds \\
6 & Phase 3 (DIL) & Fortify tomato with domain-shifted data via CONEC-LoRA, update OOD thresholds \\
7-8 & Second Crop (Pepper) & Independent Phase 1 initialization, validate standalone \\
9-10 & Third Crop (Corn) & Add third independent crop, verify no interference \\
11 & Integration & Full pipeline with routing, dynamic OOD, all crops working independently \\
12 & Demo \& Docs & Deploy Gradio demo, document system architecture and OOD statistics \\
\bottomrule
\end{tabular}
\end{table}

\section{Adapter Specification}

\begin{lstlisting}[language=Python, caption={v5.5 Adapter Configuration}]
adapter_spec_v55 = {
    # Identity
    'adapter_id': 'aads_multicrop_independent_v55',
    'architecture': 'independent_multicrop_dynamic_ood',
    
    # Crop Router Configuration (Simple)
    'crop_router': {
        'type': 'resnet50_classifier',  # Or DINOv3 linear probe
        'training_data': 'plantclef_crops',
        'target_accuracy': 0.98
    },
    
    # Per-Crop Adapter Configuration
    'per_crop': {
        'model_name': 'facebook/dinov2-giant',
        'use_dora': True,
        'lora_r': 32,
        'lora_alpha': 32,
        'loraplus_lr_ratio': 16,
        'phase1_epochs': 50,
        'phase2_epochs': 20,
        'phase3_epochs': 15
    },
    
    # OOD Detection (Dynamic Mahalanobis)
    'ood_detection': {
        'method': 'dynamic_mahalanobis',
        'threshold_factor': 2.0,  # k sigma (95% confidence)
        'min_val_samples_per_class': 10,  # Minimum for statistics
        'fallback_threshold': 25.0  # If insufficient validation data
    },
    
    # Performance Targets
    'targets': {
        'crop_routing_accuracy': 0.98,
        'phase1_accuracy': 0.95,
        'phase2_retention': 0.90,
        'phase3_retention': 0.85,
        'ood_auroc': 0.92,
        'ood_false_positive_rate': 0.05
    }
}
\end{lstlisting}

\section{Phase-by-Phase Quick Start}

\subsection{Phase 0: Crop Router Setup (Weeks 1-2)}

\begin{lstlisting}[language=bash, caption={Train Simple Crop Classifier}]
# Train simple crop classifier
python train_crop_router.py \
    --data_dir ./data/plantclef_crops/ \
    --crops tomato,pepper,corn \
    --output_dir ./models/crop_router/

# Validation: Crop classification accuracy >= 98%
python evaluate_router.py \
    --router ./models/crop_router/ \
    --test_data ./data/crops_test/
\end{lstlisting}

\subsection{Phase 1: First Crop Initialization (Weeks 3-4)}

\begin{lstlisting}[language=bash, caption={Initialize Tomato Adapter with Dynamic OOD}]
# Initialize tomato adapter with Phase 1 DoRA training
# IMPORTANT: Include validation data for OOD threshold computation
python train_phase1_crop.py \
    --crop_name tomato \
    --data_dir ./data/tomato/phase1/ \
    --val_dir ./data/tomato/val/ \
    --classes healthy,early_blight,late_blight,leaf_mold \
    --output_dir ./adapters/tomato/

# Validation: Clean accuracy >= 95%
python evaluate_crop.py \
    --adapter ./adapters/tomato/phase1/ \
    --test_data ./data/tomato/test

# Check computed OOD thresholds
python inspect_ood_stats.py \
    --adapter ./adapters/tomato/phase1/
# Output shows per-class thresholds:
#   healthy: mean=5.2, std=1.1, threshold=7.4
#   early_blight: mean=8.5, std=2.3, threshold=13.1
#   ...
\end{lstlisting}

\subsection{Phase 2: Add New Disease (Week 5)}

\begin{lstlisting}[language=bash, caption={SD-LoRA Class Increment with OOD Update}]
# Add new disease to tomato via SD-LoRA
python train_phase2_cil.py \
    --crop_name tomato \
    --phase1_checkpoint ./adapters/tomato/phase1/ \
    --new_class_data ./data/tomato/septoria_leaf_spot/ \
    --new_class_name septoria_leaf_spot \
    --output_dir ./adapters/tomato/phase2/

# Validation: Old class retention >= 90%
python evaluate_retention.py \
    --adapter ./adapters/tomato/phase2/ \
    --test_data ./data/tomato/test_all_classes

# Verify OOD thresholds updated for new class
python inspect_ood_stats.py \
    --adapter ./adapters/tomato/phase2/
# Should show new class with computed threshold
\end{lstlisting}

\subsection{Phase 3: Fortify Existing Classes (Week 6)}

\begin{lstlisting}[language=bash, caption={CONEC-LoRA Fortification with OOD Update}]
# Fortify tomato with domain-shifted data
python train_phase3_dil.py \
    --crop_name tomato \
    --phase2_checkpoint ./adapters/tomato/phase2/ \
    --fortification_data ./data/tomato/domain_shift/ \
    --target_classes early_blight,late_blight \
    --output_dir ./adapters/tomato/phase3/

# OOD thresholds automatically updated for fortified classes
python inspect_ood_stats.py \
    --adapter ./adapters/tomato/phase3/
\end{lstlisting}

\subsection{Add Second Crop Independently (Weeks 7-8)}

\begin{lstlisting}[language=bash, caption={Independent Pepper Adapter}]
# Pepper adapter - completely independent from tomato
python train_phase1_crop.py \
    --crop_name pepper \
    --data_dir ./data/pepper/phase1/ \
    --val_dir ./data/pepper/val/ \
    --classes healthy,bacterial_spot,powdery_mildew \
    --output_dir ./adapters/pepper/

# Note: No LEBA transfer, no ELLA coordination
# Each crop is self-contained with its own OOD statistics
\end{lstlisting}

\section{Production Deployment}

\begin{lstlisting}[language=Python, caption={Production Pipeline with Dynamic OOD}]
from aads_v55 import IndependentMultiCropPipeline

# Initialize pipeline
pipeline = IndependentMultiCropPipeline(config)

# Register crops independently
# OOD statistics loaded automatically
pipeline.register_crop('tomato', './adapters/tomato/')
pipeline.register_crop('pepper', './adapters/pepper/')
pipeline.register_crop('corn', './adapters/corn/')

# Production inference
while True:
    image, metadata = receive_from_sensors()
    
    # Simple routing -> adapter prediction with dynamic OOD
    result = pipeline.process_image(image, metadata)
    
    if result['action'] == 'INFERENCE':
        # Standard diagnosis with confidence and OOD info
        display_result(result)
        # result includes:
        # - disease prediction
        # - confidence score
        # - mahalanobis_distance
        # - dynamic_threshold
        # - ood_score (distance / threshold)
        
    elif result['action'] == 'TRIGGER_PHASE2':
        # New disease detected in specific crop
        send_alert(f"New disease in {result['crop']}")
        send_alert(f"OOD score: {result['ood_score']:.2f}")
        accumulate_samples(result['crop'], image)
        
    elif result['action'] == 'PHASE2_COMPLETE':
        # Crop adapter updated with new OOD thresholds
        send_alert(f"Adapter updated: {result['crop']}")
        send_alert(f"New OOD stats computed for added class")
\end{lstlisting}

\section{Success Criteria}

\begin{table}[H]
\centering
\caption{v5.5 Success Criteria}
\begin{tabular}{llc}
\toprule
\textbf{Phase} & \textbf{Metric} & \textbf{Target} \\
\midrule
Crop Router & Crop classification accuracy & $\geq$ 98\% \\
Per-Crop Phase 1 & Clean accuracy & $\geq$ 95\% \\
Per-Crop Phase 2 & Old class retention & $\geq$ 90\% \\
Per-Crop Phase 3 & Protected class retention & $\geq$ 85\% \\
OOD Detection & AUROC & $\geq$ 0.92 \\
OOD Detection & False positive rate & $\leq$ 5\% \\
System-wide & Independence & Zero interference \\
Dynamic OOD & Per-class threshold validity & All classes have stats \\
\bottomrule
\end{tabular}
\end{table}

\section{Comparison: v5.2 vs v5.3 vs v5.4 vs v5.5}

\begin{table}[H]
\centering
\caption{Architectural Comparison Across Versions}
\begin{tabular}{lcccc}
\toprule
\textbf{Feature} & \textbf{v5.2} & \textbf{v5.3} & \textbf{v5.4} & \textbf{v5.5} \\
\midrule
Crops & Single & Multi (coupled) & Multi (independent) & Multi (independent) \\
Cross-crop transfer & N/A & LEBA + ELLA & None & None \\
OOD Detection & Fixed & Meta-OOD adaptive & Fixed Mahalanobis & \textbf{Dynamic Mahalanobis} \\
Threshold type & Global & Adaptive global & Global fixed & \textbf{Per-class dynamic} \\
Manual tuning & Required & Less & Required & \textbf{Minimal} \\
Complexity & Medium & Very High & Medium & Medium \\
Implementation time & 8 weeks & 20+ weeks & 12 weeks & 12 weeks \\
Best for & Single crop & Research & Production & \textbf{Production + Robust OOD} \\
\bottomrule
\end{tabular}
\end{table}

\section{Troubleshooting}

\subsection{Crop Router: Low Accuracy}

\textbf{Symptom:} Crop classification $<$ 95\%

\textbf{Solution:}
\begin{itemize}[leftmargin=*]
    \item Use pre-trained PlantCLEF model or increase training data
    \item Ensure images contain clear leaf morphology
    \item Check for class imbalance in training data
\end{itemize}

\subsection{Phase 2 SD-LoRA: High Forgetting}

\textbf{Symptom:} Old class retention $<$ 85\%

\textbf{Solution:}
\begin{itemize}[leftmargin=*]
    \item Verify lora\_A and lora\_B are frozen
    \item Reduce learning rate for Phase 2 (try 5e-5 instead of 1e-4)
    \item Increase number of Phase 2 epochs
    \item Check that old class data loader is properly configured
\end{itemize}

\subsection{OOD Detection: Poor Calibration}

\textbf{Symptom:} Too many false positives or missed detections

\textbf{Solution:}
\begin{itemize}[leftmargin=*]
    \item \textbf{High false positives:} Increase threshold\_factor (try 2.5 for 99\% confidence)
    \item \textbf{Missed detections:} Decrease threshold\_factor (try 1.5 for 87\% confidence)
    \item Ensure sufficient validation samples per class ($\geq$ 10)
    \item Check that validation data represents true in-distribution variation
\end{itemize}

\subsection{Dynamic OOD: Missing Class Statistics}

\textbf{Symptom:} OOD stats show 0.0 for some classes

\textbf{Solution:}
\begin{itemize}[leftmargin=*]
    \item Ensure validation data has examples of all classes
    \item Check class indexing matches between training and validation
    \item Use fallback threshold (25.0) if insufficient data
\end{itemize}

\begin{thebibliography}{9}

\bibitem{liu2024}
Liu, S., et al. (2024). DoRA: Weight-Decomposed Low-Rank Adaptation. \textit{ICML 2024}.

\bibitem{wu2025}
Wu, Y., et al. (2025). SD-LoRA: Scalable Decoupled Low-Rank Adaptation for Class Incremental Learning. \textit{ICLR 2025}.

\bibitem{paeedeh2025}
Paeedeh, N., et al. (2025). Continual Knowledge Consolidation LoRA for Domain Incremental Learning. \textit{arXiv:2510.16077}.

\bibitem{lee2018}
Lee, K., et al. (2018). A Simple Unified Framework for Detecting Out-of-Distribution Samples. \textit{NeurIPS 2018}.

\bibitem{gong2024}
Gong, D., et al. (2024). Self-Expansion of Pre-trained Models with Mixture of Adapters for Continual Learning. \textit{arXiv:2403.18886}.

\end{thebibliography}

\end{document}